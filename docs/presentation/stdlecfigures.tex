% included by stdlechdr.tex

\providecommand{\twidth}{\textwidth}
\providecommand{\captionwidth}{0.9\textwidth}
\providecommand{\theight}{\textheight}

\usepackage{epsfig}

% Simplified figure macros

\newcommand{\myfreqlabel}{\latex{\small}$\omega T$ (radians per sample)}
\newcommand{\mysmallerfreqlabel}{\latex{\small}
	Frequency $\omega T$ (rad/sample)}

% Usage: \myFigure{num}{caption}
\newcommand{\myFigure}[2]{
	\begin{center}
	\epsfig{file=eps/#1.eps} \\
	  #2
	\end{center}
}

% Usage: \myFigureRotate{name}{rotationAngle}{caption}
\newcommand{\myFigureRotate}[3]{
	\begin{makeimage}
	\begin{center}
	\epsfig{file=eps/#1.eps,angle=#2} \\
	     #3
	\end{center}
	\end{makeimage}
}

% Usage: \myFigureRotateToWidth{name}{rotationAngle}{width}{caption}
\newcommand{\myFigureRotateToWidth}[4]{
	\begin{makeimage}
	\begin{center}
	\epsfig{file=eps/#1.eps,angle=#2,width=#3} \\
	     #4
	\end{center}
	\end{makeimage}
}

% Usage: \myFigureRotateTwoToWidth{name1}{name2}{rotationAngle}{width}{caption}
\newcommand{\myFigureRotateTwoToWidth}[5]{
	\begin{makeimage}
	\begin{center}
        \minipage{\twidth}
	  \epsfig{file=eps/#1.eps,angle=#3,width=#4}
	  \epsfig{file=eps/#2.eps,angle=#3,width=#4} \\
	     #5 % caption
        \endminipage
	%\resizebox{0.5 #4}{!}{\rotatebox{#3}{\includegraphics{\figdir/#1\figdotext}}}
	\end{center}
	\end{makeimage}
}

% Usage: \myFigureRotateToBox{name}{angle}{width}{height}{caption}
\newcommand{\myFigureRotateToBox}[5]{
	\begin{figure}[htbp]
	\centering
	\rotatebox{#2}{\resizebox{#4}{#3}{\includegraphics{eps/#1.eps}}} \\
	     #5 % caption
	\end{figure}
}

% FIXME: This crashes on a divide by zero the one time I tried it so far, while 'ToWidth' works fine:
% Usage: \myFigureRotateToBox{name}{angle}{width}{height}{caption}
\newcommand{\myFigureRotateToBoxFIXME}[5]{
	\begin{makeimage}
	\begin{center}
	\epsfig{file=eps/#1.eps,angle=#2,width=#3,height=#4} \\
	     #5 % caption
	\end{center}
	\end{makeimage}
}

\begin{htmlonly}
% Usage: \myFigureToWidth{num}{width}{caption}
\newcommand{\myFigureToWidth}[3]{
    \begin{makeimage}
    \begin{center}
    \epsfig{file=eps/#1.eps,width=#2} \\
	#3
    \end{center}
    \end{makeimage}
}
% Usage: \myFigureToBox{num}{width}{height}{caption}
\newcommand{\myFigureToBox}[4]{
    \begin{makeimage}
    \begin{center}
    \epsfig{file=eps/#1.eps,width=#2,height=#3} \\
	#4
    \end{center}
    \end{makeimage}
}
\end{htmlonly}

%\begin{latexonly}
% Usage: \myFigureToWidth{num}{width}{caption}
\newcommand{\myFigureToWidth}[3]{
	\begin{center}
	\epsfig{file=eps/#1.eps,width=#2} \\
	  #3
	\end{center}
}
% Usage: \myFigureToBox{num}{width}{height}{caption}
\newcommand{\myFigureToBox}[4]{
	\begin{center}
	\epsfig{file=eps/#1.eps,width=#2,height=#3} \\
	  #4
	\end{center}
}
%\end{latexonly}


% Usage: \myTexFigureNCAny{name}{extension}
% We create .picture files by exporting from xfig in .latex mode
\newcommand{\myTexFigureNCAny}[2]{
	\begin{center}
	\begin{figure}[h] % FIGURE ENV NECESSARY FOR HTML
	\centering
	\input fig/#1.#2
	\label{fig:#1} % MUST GO AFTER CAPTION OR INSIDE ITS ARGUMENT
	\end{figure}
	\end{center}
}

% Usage: \myTexFigureAny{name}{extension}{caption}
% We create .picture files by exporting from xfig in .latex mode
\newcommand{\myTexFigureAny}[3]{
	\begin{center}
	\begin{figure}[h] % FIGURE ENV NECESSARY FOR HTML
	\centering
	\input fig/#1.#2
	%\caption{#3} % Includes ``Figure N:''
	\\ {\LARGE #3}
	\label{fig:#1} % MUST GO AFTER CAPTION OR INSIDE ITS ARGUMENT
	\end{figure}
	\end{center}
}

\usepackage{epic}
\usepackage{eepic}
\usepackage{color} % xfig figures exporting pstex_t files need this
%\usepackage{eepicemu}
%\usepackage{pictex}

% Usage: \myTexFigure{name}{caption}
%\newcommand{\myTexFigure}[2]{\myTexFigureAny{#1}{eepic}{#2}} % default
%\newcommand{\myTexFigure}[2]{\myTexFigureAny{#1}{latex}{#2}} % default
% 12/16/2010/jos:
\newcommand{\myTexFigure}[2]{\myFigure{#1}{#2}} % now using figtex2eps to autocreate .eps from .fig
%\newcommand{\myTexFigure}[2]{\myTexFigureAny{#1}{pstex_t}{#2}} % default
\newcommand{\myTexFigureNC}[1]{\myTexFigureNCAny{#1}{pstex_t}} % default


% Usage: \myFigureScale{name}{scaleFactor}{caption}
\newcommand{\myFigureScale}[3]{
	\begin{center}
	\epsfbox{eps/#1.eps} \\
	     #3
	\end{center}
}

% Usage: \myFigureScaleURL{name}{scaleFactor}{caption=URL anchor}{URL}
\newcommand{\myFigureScaleURL}[4]{
	\begin{center}
	\htmlimage{map=../typical.map}
	\epsfbox{eps/#1.eps} \\
	     #3
	\end{center}
}

% Usage: \myTwoFiguresToWidth{name1}{name2}{width}{caption1}{caption2}{caption}
\newcommand{\myTwoFiguresToWidth}[6]{
	\begin{figure}[htbp]
	\centering
	\subfigure[#4]{ % Top figure = first figure
	   \epsfxsize=#3
	   \epsfbox{eps/#1.eps}
%  	   \label{fig:#1}
	}
	\subfigure[#5]{ % Bot figure = 2nd figure
	   \epsfxsize=#3
	   \epsfbox{eps/#2.eps}
%  	   \label{fig:#2}
	}
%	\caption{#6}
	\end{figure}
}

% Usage: \myTwoFiguresToBoxes{name1}{name2}{width}{sliceheight}{caption1}{caption2}{caption}
\newcommand{\myTwoFiguresToBoxes}[7]{
	\begin{figure}[htbp]
	\centering
	\subfigure[#5]{ % Top figure = first figure
	   \epsfxsize=#3
	   \epsfysize=#4
	   \epsfbox{eps/#1.eps}
  	   \label{fig:#1}
	}
	\subfigure[#6]{ % Bot figure = 2nd figure
	   \epsfxsize=#3
	   \epsfysize=#4
	   \epsfbox{eps/#2.eps}
  	   \label{fig:#2}
	}
%	\caption{#7}
	\end{figure}
}

%======================== TABLE SUPPORT ========================

% Usage: \myTable{Num[table prefix added]}{MultiLineCaption.}{table}
\newcommand{\myTable}[3]{
	\begin{table}[htbp]
	\begin{center}
	#3 \\
	\parbox{5in}{#2}
	\end{center}
	\end{table}
}

% Usage: \myTableHere{MultiLineCaption.}{table}
\newcommand{\myTableHere}[2]{
	\begin{table}[h]
	\begin{center}
	#2 \\
	\parbox{5in}{#1}
	\end{center}
	\end{table}
}

% =================== CODE LISTING SUPPORT ===================

% The code env creates a code-listing box.
%USAGE:
%  \begin{code}{name}{caption}
%  \begin{verbatim}
%  ...
%  \end{verbatim}
%  \end{code}
%
% REQUIRES:
% \usepackage{boxedminipage} % See p. 277 of LaTeX Companion

\newenvironment{code}[2]{%LATEX VERSION
\begin{figure}[htbp]
\centering
\gdef\currLabel{#1}\gdef\currCap{#2}
\setlength{\fboxsep}{3mm}
\begin{boxedminipage}{\captionwidth}
\small  % 9/23/03
}%
{
\end{boxedminipage}
\parbox{\captionwidth}{{\small\caption{\currCap}\label{code:\currLabel}}}
\end{figure}
}
