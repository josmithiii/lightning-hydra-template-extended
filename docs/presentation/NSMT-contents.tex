\section[toc={Neural Spectral Modeling}]{Neural Spectral Modeling Template (NSMT)}
  
\begin{emptyslide}[toc={}]{}
\vspace{-1.44em}
\myFigureRotateToBox{ADCxGather-2025-title-part2}{-90}{\twidth}{\theight}{}
%\myFigureRotateToBox{ADCxGather-2025-title-NSMT}{-90}{\twidth}{\theight}{}
\vspace{\stretch{1}}
\end{emptyslide}

\begin{slide}[\slideopts,toc={NSMT Overview}]{Neural Spectral Modeling Template (NSMT) --- Spectral-First Audio Processing}

  \vspace{-1em}

  \begin{itemize}

    \mpitem NSMT is a fork of \emph{Lightning Hydra Template Extended (LHTE)} adapted specifically for
    \textbf{audio spectrograms}

    \mpitem \emph{Key Philosophy:} All ``images'' are audio \emph{spectral representations}:
    \vspace{-0.5em}
    \begin{itemize}
      \mpitem Image \emph{height} = frequency bins
      \mpitem Image \emph{width} = time frames
      \mpitem Image \emph{channels} = alternate spectral representations:
      \begin{itemize}
        \mpitem Alternate time-frequency resolutions
        \mpitem Instantaneous-frequency / group-delay maps
        \mpitem Modulation spectra, Etc.
      \end{itemize}
    \end{itemize}

    \vspace{-0.5em}
    \mpitem \emph{Goal:} Enable \emph{small, accurate, and fast} neural networks for audio processing
    \vspace{-0.5em}
    \begin{itemize}
      \mpitem Accept inductive priors of human hearing
      \mpitem Use efficient CNN architectures with conditioning inputs
      \mpitem Avoid expensive "end-to-end" approaches that require massive data/compute
    \end{itemize}

    \vspace{-0.5em}
    \mpitem Uses only VIMH dataset format from LHTE (drops MNIST/CIFAR support)
    
    \mpitem Forked from LHTE commit \texttt{3795aad} on August 26, 2024

  \end{itemize}

\end{slide}

\begin{slide}[\slideopts,toc={Spectra}]{Why Spectral Representations?}

  Aren't we supposed to be doing everything \emph{end to end} now?

  Shouldn't the input be a digital audio \emph{bit stream} by now?

  \begin{itemize}

    \mpitem Sure, you can do it that way, but with \emph{far more}
    \begin{itemize}
      \mpitem training \emph{data},
      \mpitem training \emph{time}, 
      \mpitem computational complexity (e.g., Transformer)
    \end{itemize}
    \mpitem It is more cost-effective to exploit \emph{inductive priors:}
    \begin{itemize}
      \mpitem The \emph{ear} is a \emph{hardware spectrum analyzer} used for all audio perception
      \mpitem \emph{Superhuman hearing} is possible using a \emph{stack of different time-frequency resolutions}\\
      (Multi-Scale Spectrograms)
      \mpitem Any additional feature can be brought in as a \emph{conditioning} input\\
      (such as \emph{pre-emphasis})
    \end{itemize}

  \end{itemize}

  \vspace{-1em}
%  \begin{quote}
\maybepause
    \centerline{\textit{Those who don't know signal processing are doomed to reinvent it}}
\maybepause
    \centerline{\textit{Those who know signal processing are doomed to re-introduce it}}
%  \end{quote}

\end{slide}

\begin{wideslide}[\slideopts,toc={McDermott}]{McDermott Lab Example: Kell et al., Neuron 2018}
  %\vspace{-2.5em}
  %% ``A Task-Optimized Neural Network Replicates Human Auditory
  %% Behavior, Predicts Brain Responses, and Reveals a Cortical
  %% Processing Hierarchy''
  %% \url{https://doi.org/10.1016/j.neuron.2018.03.044}
  %% Josh McDermott's Computational Cognitive Neuroscience lab
%  \url{http://mcdermottlab.mit.edu/papers/Kell_etal_2018_DNN_auditory_cortex.pdf}
  \twocolumn{
  \vspace{-2em}
  \myFigureRotateToWidth{McDermottKell2018}{-90}{\twidth}{}
  }{
  \vspace{-2em}
  \begin{itemize}
    \mpitem \emph{Cochleagram} presented as an image input to the convolutional neural network (CNN)
    \mpitem For word recognition and music genre classification, this 12-layer network emerged as best (out of 180 tried)
    \mpitem The 1st 7 \emph{shared} layers were found to perform comparably with separate 12-layer networks
    \mpitem Performs at human level, with similar errors
    \mpitem Predicts fMRI responses in the audio cortex
    \mpitem Outperforms previous spectrotemporal filter models of the audio cortex
    \mpitem Cochleagram required for predicting human responses\\
    \mpitem Perhaps the cochleagram can be replaced by vastly more natural data and training (evolution-equivalent)
    %% % Easy to see from figure:
    %% \begin{itemize}
    %% \item five convolutional layers
    %% \item three pooling
    %% \item two normalization
    %% \item two fully connected
    %% \end{itemize}
  \end{itemize}
}
\end{wideslide}

%% \begin{slidewhite}[\slideopts,toc={Problem}]{Example Driving Problem: Real-Time Filter Design in an Audio Plugin}
%%   \vspace{-2em}
%%   \myFigureRotateToWidth{pgmeejos}{-90}{0.85\twidth}{(Red-Bordered Buttons Added to
%%     \textbf{Plugin GUI Magic}'s Equalizer Example)}
%% \end{slidewhite}

%% Let xxx denote the architecture, e.g., vit_tiny
%% \begin{slidewhite}[\slideopts,toc={xxx}]{Architecture xxx}
%% \vspace{-2em}
%% \myFigureToWidth{fileNameWithoutSuffix}{1.0\twidth}{}
%% \end{slidewhite}

%% % Let xxx denote the architecture, e.g., vit_tiny
%% \begin{slidewhite}[\slideopts,toc={xxx}]{Architecture xxx}
%% \vspace{-2em}
%% \myFigureToWidth{fileNameWithoutSuffix}{1.0\twidth}{}
%% \end{slidewhite}

\begin{slide}[\slideopts,toc={VIMH Format}]{VIMH Dataset Format --- Self-Describing, Multihead}
  \vspace{-0.75em}
  \begin{itemize}
    \mpitem \textbf{VIMH} = Variable Image MultiHead format for spectrogram-like images
    \vspace{-0.25em}
    \begin{itemize}
      \mpitem Stores H, W, C + variable-length label list per sample
      \mpitem Supports 1--255 quantized continuous parameters (0--255)
      \mpitem Single dataset auto-configures model input + heads
    \end{itemize}
    \vspace{0.25em}
    \mpitem \textbf{Label encoding:} [N, id\_1, val\_1, id\_2, val\_2, \dots]
  \end{itemize}
  \vspace{-0.5em}
  \begin{center}
    %1 Old obscure vimh example: \includegraphics[width=0.78\linewidth]{docs/png/sample_data.eps}
    %2 Better; \myFigureRotateToBox{vimh-proj}{-90}{\twidth}{0.45\theight}{}
    %3 \vspace{-4em}
    %3 \myFigureRotateToWidth{McDermottKell2018}{-90}{\twidth}{}
    %4
    %\myFigureRotateToBox{McDermottKell2018}{-90}{\twidth}{0.45\theight}{}
    % CNN diagram for NSMT (dvips-friendly)
    % Requires:
    %   \usepackage{tikz}
    %   \usetikzlibrary{positioning}

\begin{tikzpicture}[x=1mm,y=1mm, line join=round, line cap=round]

  % pseudo-3D offsets
  \def\depthx{3}
  \def\depthy{1.2}

  % Draw a 3D panel (front + right face)
  % Args: optional label, x, y, width, height, front fill color
  \newcommand{\panel}[6][]{%
    \begin{scope}[shift={(#2,#3)}]
      % front
      \filldraw[fill=#6, draw=black] (0,0) rectangle (#4,#5);
      % right face (fixed shade to avoid nested color mix errors)
      \filldraw[fill=black!20, draw=black]
        (#4,0) -- ++(\depthx,\depthy) -- ++(0,#5) -- ++(-\depthx,-\depthy) -- cycle;
      % label (optional)
      \ifx\\#1\\\else
        \node[font=\scriptsize, anchor=north] at (#4*0.5,-1.2) {#1};
      \fi
    \end{scope}%
  }

  % Spectrogram-like panel (striped front)
  % Args: optional label, x, y, width, height, front fill color
  \newcommand{\spectro}[6][]{%
    \begin{scope}[shift={(#2,#3)}]
      % front
      \filldraw[fill=#6, draw=black] (0,0) rectangle (#4,#5);
      % vertical stripes
      \foreach \xx in {2,4,6,8,10,12,14,16,18}{\draw[gray!60] (\xx,0) -- (\xx,#5);} 
      % horizontal bands
      \foreach \yy in {3,6,9,12,15,18,21,24}{\draw[gray!30] (0,\yy) -- (#4,\yy);} 
      % right face
      \filldraw[fill=black!20, draw=black]
        (#4,0) -- ++(\depthx,\depthy) -- ++(0,#5) -- ++(-\depthx,-\depthy) -- cycle;
      % label
      \ifx\\#1\\\else
        \node[font=\scriptsize, anchor=north] at (#4*0.5,-1.2) {#1};
      \fi
    \end{scope}%
  }

 

  % Positions
  \def\xin{0}  \def\yin{0}
  \def\xbA{40} \def\ybA{-2}
  \def\xbB{72} \def\ybB{0}
  \def\xbC{98} \def\ybC{6}
  \def\xbD{118}\def\ybD{10}

  % Input: 4 stacked spectrograms
  \spectro[32$\times$32$\times$1]{\xin}{\yin}{22}{26}{blue!12}
  \spectro[]{\xin-2}{\yin+1.5}{22}{26}{blue!10}
  \spectro[]{\xin-4}{\yin+3.0}{22}{26}{blue!8}
  \spectro[]{\xin-6}{\yin+4.5}{22}{26}{blue!6}
  \node[font=\scriptsize] at (\xin+7, \yin+30) {4 stacked spectrograms};

  % Connector to Conv1
  \filldraw[fill=gray!50, draw=black] (26,10) rectangle (38,16);

  % Conv1 (16 channels)
  \panel[\scriptsize Conv1: 16 ch]{\xbA}{\ybA}{18}{32}{orange!20}
  \node[font=\scriptsize, anchor=south] at (\xbA+9,\ybA+32+2) {3$\times$3, BN, ReLU};

  % Connector to Conv2
  \filldraw[fill=gray!50, draw=black] (\xbA+18+\depthx,\ybA+12+\depthy) rectangle (\xbB-2,\ybB+18);

  % Conv2 (32 channels)
  \panel[\scriptsize Conv2: 32 ch]{\xbB}{\ybB}{16}{28}{green!20}
  \node[font=\scriptsize, anchor=south] at (\xbB+8,\ybB+28+2) {3$\times$3, BN, ReLU};
  \node[font=\scriptsize, anchor=north] at (\xbB+8,\ybB-2) {MaxPool};

  % Connector to FC
  \filldraw[fill=gray!50, draw=black] (\xbB+16+\depthx,\ybB+12+\depthy) rectangle (\xbC-2,\ybC+18);

  % FC (64)
  \panel[\scriptsize FC: 64]{\xbC}{\ybC}{12}{20}{violet!20}
  \node[font=\scriptsize, anchor=south] at (\xbC+6,\ybC+20+2) {Dropout 0.3};

  % Connector to heads
  \filldraw[fill=gray!50, draw=black] (\xbC+12+\depthx,\ybC+9+\depthy) rectangle (\xbD-2,\ybD+14);

  % Regression heads (3 example heads)
  \panel[\scriptsize Head 1]{\xbD}{\ybD}{6}{14}{gray!20}
  \panel[\scriptsize Head 2]{\xbD+8}{\ybD-1}{6}{12}{gray!20}
  \panel[\scriptsize Head 3]{\xbD+16}{\ybD-2}{6}{10}{gray!20}
  \node[font=\scriptsize, anchor=south] at (\xbD+9,\ybD+16) {Regression heads};

\end{tikzpicture}
    
\end{center}
\end{slide}

\begin{slide}[\slideopts,toc={Losses}]{Distance-Aware Losses for Quantized Parameters}
  \vspace{-0.75em}
  \begin{itemize}
    \mpitem Challenge: 8-bit quantization (0--255) of continuous parameters
    \mpitem \textbf{OrdinalRegressionLoss} $\to$ continuous, distance-aware predictions
    \mpitem \textbf{QuantizedRegressionLoss} $\to$ direct regression on quantized scale
    \mpitem \textbf{WeightedCrossEntropyLoss} $\to$ classification with distance penalties
  \end{itemize}
  \vspace{-0.5em}
  \begin{center}
    \includegraphics[width=0.45\linewidth]{docs/png/vimh_loss_analysis.eps}\hspace{0.02\linewidth}
    \includegraphics[width=0.45\linewidth]{docs/png/vimh_perceptual_loss_analysis.eps}
\end{center}
\end{slide}

\begin{slide}[\slideopts,toc={Models}]{Model Families — Small, Accurate, and Fast}
  \vspace{-0.75em}
  \begin{itemize}
    \mpitem \textbf{CNN variants:} \texttt{cnn\_micro}, \texttt{cnn\_tiny}, \texttt{cnn\_64k}
    \vspace{-0.25em}
    \begin{itemize}
      \mpitem Multihead heads auto-wired from VIMH metadata
      \mpitem Options: \texttt{cnn\_64k\_ordinal}, \texttt{cnn\_64k\_regression}, \texttt{cnn\_64k\_auxiliary}
    \end{itemize}
    \mpitem \textbf{ViT variants:} \texttt{vit\_micro}, \texttt{vit\_tiny} (lightweight, patch-based)
    \mpitem \textbf{Design goal:} practical models that train fast, generalize well
  \end{itemize}
  \vspace{0.25em}
  \begin{center}
    \begin{minipage}{0.95\linewidth}
      \centering
      \setlength{\fboxsep}{4pt}\fbox{\parbox{0.9\linewidth}{
          CNN: Conv $\to$ Conv $\to$ Pool $\to$ MLP $\to$ \textit{heads} (1/head) \hfill
          ViT: Patches $\to$ Transformer Blocks $\to$ \textit{heads}
      }}
    \end{minipage}
  \end{center}
\end{slide}

\begin{slide}[\slideopts,toc={Auto-Config}]{Auto-Configuration from Data to Model}
  \vspace{-0.75em}
  \begin{itemize}
    \mpitem \textbf{One dataset $\to$ complete model wiring:}
    \vspace{-0.25em}
    \begin{itemize}
      \mpitem Read \texttt{vimh\_dataset\_info.json} $\to$ infer H, W, C, head names, ranges
      \mpitem DataModule passes shapes + parameter ranges to module
      \mpitem \texttt{VIMHLitModule} configures heads, losses, and metrics
    \end{itemize}
  \end{itemize}
  \vspace{-0.25em}
  \begin{center}
    \begin{tikzpicture}[>=stealth, node distance=1.8cm]
      \tikzstyle{box}=[draw, rounded corners, fill=blue!7, inner sep=6pt]
      \node[box] (json) {VIMH JSON (H,W,C, names, ranges)};
      \node[box, right=of json] (dm) {DataModule (transforms, loaders)};
      \node[box, right=of dm] (lit) {VIMHLitModule (auto-heads, losses)};
      \node[box, right=of lit] (train) {Trainer (callbacks, logs)};
      \draw[->] (json) -- (dm);
      \draw[->] (dm) -- (lit);
      \draw[->] (lit) -- (train);
    \end{tikzpicture}
  \end{center}
\end{slide}

\begin{slide}[\slideopts,toc={Workflow}]{Hands-On Workflow --- Datasets, Training, Diagrams}
  \vspace{-0.75em}
  \begin{itemize}
    \mpitem \textbf{Datasets}
    \vspace{-0.25em}
    \begin{itemize}
      \mpitem \texttt{make sds} (256) \quad \texttt{make sdl} (16k) \quad \texttt{make ddr|dds|ddl}
    \end{itemize}
    \mpitem \textbf{Training}
    \vspace{-0.25em}
    \begin{itemize}
      \mpitem \texttt{make trq} (quick sanity) \quad \texttt{make tr} (defaults)
      \mpitem \texttt{python src/train.py model=cnn\_64k\_ordinal trainer=mps}
    \end{itemize}
    \mpitem \textbf{Diagrams and Logs}
    \vspace{-0.25em}
    \begin{itemize}
      \mpitem \texttt{make tds|td|tdsa} (model diagrams) \quad \texttt{make tensorboard}
    \end{itemize}
  \end{itemize}
  \vspace{0.25em}
  \begin{center}
    \includegraphics[width=0.7\linewidth]{docs/png/parameter\_distributions.eps}
  \end{center}
\end{slide}

\begin{slide}[\slideopts,toc={Audio Eval}]{Audio Reconstruction Evaluation --- Hear the Difference}
  \vspace{-0.75em}
  \begin{itemize}
    \mpitem \texttt{python src/audio\_reconstruction\_eval.py} (or \texttt{make ae})
    \mpitem Auto-discovers latest checkpoint, launches interactive widget
    \mpitem Compares audio synthesized from true vs predicted parameters
    \mpitem Metrics: SNR, correlation, RMSE + parameter errors
  \end{itemize}
  \vspace{0.25em}
  \begin{center}
    \setlength{\fboxsep}{6pt}\fbox{\parbox{0.88\linewidth}{
        spectrogram $\to$ model $\to$ predicted params $\to$ \textit{synth} $\to$ audio $\leftrightarrow$ compare with ground truth
    }}
  \end{center}
\end{slide}

\begin{slide}[\slideopts,toc={Moog VCF}]{Moog VCF Experiments --- Basic, Envelope, Resonance}
  \vspace{-0.75em}
  \begin{itemize}
    \mpitem \textbf{Datasets}
    \vspace{-0.25em}
    \begin{itemize}
      \mpitem \texttt{sdmb} (4 params), \texttt{sdme} (10 params), \texttt{sdmr} (8 params)
    \end{itemize}
    \mpitem \textbf{Training}
    \vspace{-0.25em}
    \begin{itemize}
      \mpitem \texttt{emb}, \texttt{eme} (ordinal and regression), \texttt{emr}; ViT variants via \texttt{emvit*}
    \end{itemize}
    \mpitem \textbf{Purpose:} control complexity, probe generalization across parameter spaces
  \end{itemize}
  \vspace{0.25em}
  \begin{center}
    \setlength{\fboxsep}{4pt}\fbox{\parbox{0.9\linewidth}{
        Color map: parameter sweeps across time $\to$ spectral patterns $\to$ multihead predictions
    }}
  \end{center}
\end{slide}

\begin{slide}[\slideopts,toc={Reproducibility}]{Reproducibility, Configs, and Takeaways}
  \vspace{-0.75em}
  \begin{itemize}
    \mpitem \textbf{Hydra:} \texttt{configs/train.yaml} + \texttt{experiment=example}
    \mpitem \textbf{Preflight checks:} label diversity, dataset introspection
    \mpitem \textbf{Tags + Logs:} group runs, compare configs in TensorBoard
    \mpitem \textbf{Key outcomes}
    \vspace{-0.25em}
    \begin{itemize}
      \mpitem Distance-aware losses $\to$ perceptually meaningful errors
      \mpitem VIMH auto-config $\to$ less boilerplate, fewer mistakes
      \mpitem Small models $\to$ faster iteration; competitive accuracy
    \end{itemize}
  \end{itemize}
  \vspace{0.25em}
  \begin{center}
    \includegraphics[width=0.65\linewidth]{docs/png/confusion_matrices.eps}
  \end{center}
\end{slide}
