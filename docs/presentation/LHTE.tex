% SHORT VERSION FOR ASA (12 min)
%&latex

\newcommand{\theTitle}{Python Templates for Neural Image Classification and Spectral Audio Processing}
\newcommand{\theSubTitle}{Lightning Hydra Template Extended}
%\newcommand{\theSubTitle}{Neural Spectral Modeling Template}
\newcommand{\theShortTitle}{LHTE}
\newcommand{\theEvent}{Audio Developer Conference Online (ADCxGather-25)}
\newcommand{\theShortEvent}{ADCxGather-25}
\newcommand{\theDate}{September 26, 2025}

\input stdpreshdr.tex
% WEAN: \input ../latex/wgtmac.tex
\usepackage{xcolor}
\usepackage{psfrag}
\usepackage{graphicx}
\usepackage{etoolbox} % for \newtoggle et al
\newtoggle{paper}
\togglefalse{paper}
\usepackage{amsmath, amssymb}
\usepackage[most]{tcolorbox}
\usepackage{booktabs}
\usepackage{tikz,pgfplots}
\pgfplotsset{compat=1.18}
\usetikzlibrary{arrows.meta, positioning}

\usepackage{makecell}

\iftoggle{paper}{%
\message{*** Using IEEE Reference format ***}
\bibliographystyle{ieeetr} % Normal IEEE Transactions (in order of citation)
}{
}

\newcommand{\theAuthor}{Julius O. Smith III}
\newcommand{\theShortAuthor}{JOS}

%\renewcommand{\mpause}{\pause} % maybe pause
%\renewcommand{\mpause}{} % or not
%\newcommand{\pitem}{\mpause\item}

%HANDOUTS (includes \begin{note} envs - about 2 pages per page):
%!\documentclass[style=fymajos,mode=handout,display=slidesnotes]{powerdot}

\newtoggle{local}
%\toggletrue{local} % presentation mode
\togglefalse{local} % handout mode
\input localremote.tex

\date{\theDate}

% Notation:
\input macros.tex

\newtoggle{asa25}
\toggletrue{asa25}

%\toggletrue{paper}
\togglefalse{paper}
\iftoggle{paper}{%
  \message{*** COMPILING PAPER ***}
  \newcommand{\papersection}{\section}
  \newcommand{\mycite}{\cite} % normal LaTeX (for paper)
}{
  \message{*** COMPILING PRESENTATION ***}
  \newcommand{\papersection}[2]{\begin{slide}{#1}#2\end{slide}}
  \newcommand{\papersectionalt}[3]{\begin{slide}[toc=#1]{#2}#3\end{slide}}
  \newcommand{\papersectionaltalt}[3]{\begin{slide}[toc=#2]{#1}#3\end{slide}}
  \newcommand{\papersectionwide}[2]{\begin{wideslide}{#1}#2\end{wideslide}}
  \newcommand{\papersectionwidewhite}[2]{\begin{wideslidewhite}{#1}#2\end{wideslidewhite}}
  \newcommand{\papersectionHidden}[2]{\begin{slide}{#1}#2\end{slide}}
  \newcommand{\mycite}{} % Disable citations for presentation
}

\title{\theTitle}

\subtitle{\theSubTitle}

\author{\theAuthor\\
CCRMA, Stanford University \\[10pt]
\theEvent
}

\begin{document}

\maketitle

% ~8-10 slides planned:

\input LHTE-contents.tex
%\input NSMT-contents.tex

%% % FIGURE EXAMPLE

%% \begin{slide}[\slideopts,toc={CCRMA}]{\CCRMA\ occupies The Knoll on Stanford Campus}
%%   \vspace{-2em}
%%   \myFigureRotateToWidth{knoll}{-90}{0.7\twidth}{}
%%   \centerline{(Originally home of the Stanford University President)}
%%   \centerline{Stanford Campus Map}
%% \end{slide}

%% % SAME FIGURE WITH WHITE BACKGROUND

%% \begin{slidewhite}[\slideopts,toc={CCRMA}]{\CCRMA\ occupies The Knoll on Stanford Campus}
%%   \vspace{-2em}
%%   \myFigureRotateToWidth{knoll}{-90}{0.7\twidth}{}
%%   \centerline{(Originally home of the Stanford University President)}
%%   \centerline{Stanford Campus Map}
%% \end{slidewhite}

%% % WIDE FIGURE EXAMPLE

%% \begin{wideslide}[\slideopts,toc={McDermott}]{McDermott Lab Example: Kell et al., Neuron 2018}
%%   \twocolumn{
%%   \vspace{-2em}
%%   \myFigureRotateToWidth{McDermottKell2018}{-90}{\twidth}{}
%%   }{
%%   \vspace{-2em}
%%   \begin{itemize}
%%     \mpitem \emph{Cochleagram} presented as an image input to the convolutional neural network (CNN)
%%     \mpitem For word recognition and music genre classification, this 12-layer network emerged as best (out of 180 tried)
%%     \mpitem The 1st 7 \emph{shared} layers were found to perform comparably with separate 12-layer networks
%%     \mpitem Performs at human level, with similar errors
%%     \mpitem Predicts fMRI responses in the audio cortex
%%     \mpitem Outperforms previous spectrotemporal filter models of the audio cortex
%%     \mpitem Cochleagram required for predicting human responses\\
%%     \mpitem Perhaps the cochleagram can be replaced by vastly more natural data and training (evolution-equivalent)
%%   \end{itemize}
%% }
%% \end{wideslide}

%% \papersection{References}{
%%   \BIT
%%   \item
%%   \item
%%   \item
%%   \EIT
%% }

%\bibliographystyle{ieeetr} % Already set in stdprescommon.tex

% This does not work to split the references onto multiple successive slides - they all come out on one page:
%% \papersection{References (2018-2021)}{
%% \nocite{YeeKing2018,Barkan2019,Esling2019,MasudaSaito2021,Martinez2021}
%% \bibliography{jos,ASA-2025-SynthMatch}
%% }
%% \papersection{References (2022-2024)}{
%% \nocite{Steinmetz2022,Chen2022,MasudaSaito2023,Uzrad2024,Bruford2024}
%% \bibliography{ASA-2025-SynthMatch}
%% }

\iftoggle{paper}{%
  \nocite{*}
  \bibliography{ASA-2025-SynthMatch}
}{
  \message{*** REFERENCES SUPPRESSED FOR POWERDOT PRESENTATION ***}
  % But if you really want some:
  %% \papersection{References}{
  %%   \nocite{YeeKing2018,Bruford2024}
  %%   \bibliography{ASA-2025-SynthMatch}
  %% }
}

\end{document}

